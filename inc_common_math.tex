% math stuff
\newcommand{\eqv}{\Leftrightarrow}

\newcommand{\pvec}[1]{\ensuremath{\overrightarrow{#1}}}
\newcommand{\npvec}[1]{\ensuremath{\norm{\pvec{#1}}}}

\newcommand{\mb}[1]{\ensuremath{\mathbb{#1}}}
\newcommand{\mc}[1]{\ensuremath{\mathcal{#1}}}

\providecommand{\abs}[1]{\ensuremath{  \left \lvert #1 \right \rvert } }
\providecommand{\norm}[1]{\ensuremath{ \left \lVert #1 \right \rVert } }

\providecommand{\pth}[1]{\ensuremath{  \left ( #1 \right ) } }
\providecommand{\spth}[1]{\ensuremath{ \left [ #1 \right ] } }
\providecommand{\cpth}[1]{\ensuremath{ \left \{ #1 \right \} } }

%
% math hacks
%

% scalable middle bar; see 
% http://tex.stackexchange.com/questions/5502/how-to-get-a-mid-binary-relation-that-grows
\newcommand{\relmiddle}[1]{\mathrel{}\middle#1\mathrel{}}

% it's a cheat but this makes spacing less wrong around quantifiers
\DeclareMathOperator{\Nexists}{\nexists}
\DeclareMathOperator{\Exists}{\exists}
\DeclareMathOperator{\Forall}{\forall}
