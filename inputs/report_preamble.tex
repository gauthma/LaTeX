%
% Load ALL Packages. Package loading order matters A LOT, and so it is very
% dangerous to separate package loading into several different (\input'd)
% files. It will likely make troubleshooting of errors far more difficult. So
% just load all packages in the inputs/packages.tex file.
%
\input{inputs/packages}

% -------------------------------------------------------
% -------------------------------------------------------
% ------------------- CUSTOM STUFF ----------------------
% -------------------------------------------------------
% -------------------------------------------------------

% Use this space for custom things, other than \usepackage's. If the
% customisations are lengthy, consider putting them into their own file, to be
% \input'd. Such files should go in the inputs/ folder.

% Set up FONTS to use.

% Customise \mathcal.
\DeclareSymbolFont{usualmathcal}{OMS}{cmsy}{m}{n}
\DeclareSymbolFontAlphabet{\mathcal}{usualmathcal}

% Customise \mathbb.
\DeclareSymbolFont{mpazo}{U}{fplmbb}{m}{n}
\DeclareSymbolFontAlphabet{\mathbb}{mpazo}

\setmainfont[%
  Path=$HOME/.fonts/truetype/ ,
  Extension      = .ttf       ,
  UprightFont    = *-R        ,
  ItalicFont     = *-I        ,
  BoldFont       = *-B        ,
  BoldItalicFont = *-BI       ,
  Scale=0.9 ,
]{CharisSIL}
\defaultfontfeatures[CharisSIL]{Script=latn, Ligatures=TeX}


\input{inputs/style}

\newcommand*{\emd}{\textemdash}
\newcommand{\fn}[1]{\footnote{#1}}
\newcommand*{\namedparagraph}[1]{\noindent\textbf{#1}}
\newcommand*{\noindentparagraph}{\noindent}
\newcommand*{\ts}{\textsection}

% -------------------------------------------------------
% -------------------------------------------------------
% ------------------- END CUSTOM STUFF ------------------
% -------------------------------------------------------
% -------------------------------------------------------

% Language and bibliography. Language with polyglossia must be set before
% biblatex is used, or error ensues. But it's probably good advice with babel
% and bibtex also.
% For Portuguese, replace "UKenglish" with "portuges".
\usepackage[UKenglish]{babel}
\usepackage[comma, numbers]{natbib}
\renewcommand*{\bibsection}{\chapter*{References}}  % Name the bib chapter References.
\renewcommand*{\bibfont}{\footnotesize} % Use the same font size as in footnotes.
\setlength{\bibsep}{0pt plus 0.5ex} % Smaller vertical separation between bibtems.

% Make bib listing with <number><dot>.
\makeatletter
\renewcommand\@biblabel[1]{\textrm{#1.}}
\makeatother
% \renewcommand{\UrlFont}{\small\tt}

% hyperref et al.
% ***NOTA BENE:*** The hyperref package if used, MUST BE THE LAST ONE included!
%
\usepackage{xcolor} % For MidnightBlue colour!
\providecolors{MidnightBlue}
\usepackage[bookmarks=true,
            citecolor=MidnightBlue,
            colorlinks=true,
            hyperfootnotes=false,
            linkcolor=MidnightBlue,
            linktocpage=true,
            linktoc=all,
            pagebackref=true,
            urlcolor=MidnightBlue]{hyperref}
\renewcommand*{\backref}[1]{}
\renewcommand*{\backrefalt}[4]{%
  \ifcase #1 Not cited.%
    \or Cited on p.~#2.%
    \else%
      \ifnum #1<10 Cited on pp.~#2.%
      \else Cited #3 times.\fi%
  \fi}
