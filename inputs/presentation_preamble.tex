% Setup authorship. This is placed in the preamble because it is not supposed
% to change much...
% NOTA BENE: Use the \inst{?} command only if the authors have different
% affiliation.
\author
{Óscar Pereira\inst{1,2} \and John Doe\inst{2}\\[0.75em]
  \texttt{\scriptsize \{https://, oscar@\}randomwalk.eu}\\
  \texttt{\scriptsize \{https://, john@\}somedoe.com} 
}
\institute
{
  \inst{1}%
  Some Institution
  \and
  \inst{2}%
  Some other Institution
}

% Fonts.
\usepackage{times}
\usepackage{fontspec}
\defaultfontfeatures{Ligatures=TeX}

% Language: replace UKenglish with portuguese or other lang.
\usepackage[UKenglish]{babel}

\usepackage{sfmath} % Math in sans serif font.
\usepackage{tikz} % To draw pictures!
\usepackage{textpos} % To draw in absolute positions.
\usepackage{marvosym} % For \Smiley, and LOTS of other things

% In slides, emphasis means italic AND bold face.
\renewcommand{\emph}[1]{\textbf{\textit{#1}}}

% For bibliography: in presentations, [author, year] is more meaningful then
% [number].
\usepackage[authoryear]{natbib} 

% Gets rid of bottom navigation symbols.
\setbeamertemplate{navigation symbols}{}

% Show page numbers.
\setbeamertemplate{footline}[frame number]

% Setup theme, colours, and transparency use.
\mode<presentation>
{
  \usetheme{default}
  \usecolortheme{whale}

  % for transparencies
  \setbeamercovered{transparent}
}

% If you have a file called "university-logo-filename.xxx", where xxx
% is a graphic format that can be processed by latex or pdflatex,
% resp., then you can add a logo as follows:

% \pgfdeclareimage[height=0.5cm]{university-logo}{university-logo-filename}
% \logo{\pgfuseimage{university-logo}}

% -------------------------------------------------------
% -------------------------------------------------------
% ------------------- CUSTOM STUFF ----------------------
% -------------------------------------------------------
% -------------------------------------------------------

% Use this space for custom things, other than \usepackage's. If the
% customisations are lengthy, consider putting them into their own file, to be
% \input'd. Such files should go in the inputs/ folder.

\newcommand{\emd}{\textemdash}
\newcommand{\ts}{\textsection}

% -------------------------------------------------------
% -------------------------------------------------------
% ------------------- END CUSTOM STUFF ------------------
% -------------------------------------------------------
% -------------------------------------------------------
