% My CV's .tex file. Inspired by this one:
% http://www.danielmathews.info/about-me
%
% A pdf version (probably with outdated information, but with the
% same style) can be found here:
% https://erroneousthoughts.org/media/OscarPereiraCV.pdf
%
% WARNING: as it uses custom fonts, this file is supposed to be
% compiled with Lua(La)TeX.
% 
% NOTA BENE: this is an OUTDATED version of my CV, to illustrate
% the LaTeX structure only.
%
\documentclass[a4paper,10pt]{extarticle}

\usepackage[top=2.5cm,left=3cm,right=3cm,bottom=2.5cm]{geometry}

\usepackage[bitstream-charter]{mathdesign}
\DeclareSymbolFont{usualmathcal}{OMS}{cmsy}{m}{n}
\DeclareSymbolFontAlphabet{\mathcal}{usualmathcal}
\usepackage{fontspec}
\setmainfont[Scale=0.9, Ligatures=TeX]{Charis SIL}
\defaultfontfeatures[Charis SIL]{Script=latn}

% Languages et al.
\usepackage[UKenglish]{babel}

% This improves the quality of the PDF, but slows down the creation
% time, so you might want to comment it until you have the final version.
\usepackage{microtype}

\usepackage{enumitem}
% The fontawesome package requires (with XeLaTeX) setting the below path
% manually. It is the folder that contains the file FontAwesome.otf.
\defaultfontfeatures{
    Path = /usr/share/texmf-dist/fonts/opentype/public/fontawesome/ }
\usepackage{fontawesome}  % For the cute home, email, etc, symbols.
\usepackage{url}
\usepackage[usenames,dvipsnames]{xcolor}  % for MidnightBlue colour
\usepackage[colorlinks=true, urlcolor=MidnightBlue]{hyperref}

\renewcommand\labelitemi{\fontfamily{lmr}\selectfont\textbullet}

\newenvironment{topic}[1]
   {{\noindent\large\bfseries\raisebox{0pt}[\height][1ex]{#1}\hrule\par}%
    \begin{list}{}{%
       \setlength{\leftmargin}{.0cm}}\item[]}
   {\end{list}\medskip}

\newenvironment{header}[1]
   {\raggedleft{\noindent\large\bfseries\raisebox{0pt}[\height][1ex]{%
     \Large\bfseries{\itshape Curriculum Vit\ae}\ \hspace{0.1em} --- \hspace{0.1em} #1\par}\hrule}%
		 \begin{center}}
		 {\end{center}\medskip}

\pagestyle{empty}

\begin{document}

\begin{header}{Óscar Francisco Godinho Pereira}
  \faHome \enskip  Postal address and phone number have been withheld for privacy reasons\\[0.7em]
  \faFirefox \enskip \url{https://erroneousthoughts.org}%
    \enskip \textbullet \enskip%
    \faEnvelopeO \enskip \texttt{oscar@domain.name}%
    \enskip \textbullet \enskip%
    \faPhone\enskip +0 00000
\end{header}

\begin{topic}{Formal Education}
  \begin{itemize}[leftmargin=*]
    \item {\bfseries University of Minho}, Braga, Portugal.\\
      PhD in Informatics, MAP-i Doctoral Programme, Fall 2015--present.
    \item {\bfseries University of Minho}, Braga, Portugal.\\
			MSc in Informatics Engineering, Fall 2013--Fall 2015. \emph{Grade:} 16 out 
			of 20.\\
			\emph{Master Thesis}: Towards a Fully Algebrisable Symmetric Cryptosystem. 
			\emph{Grade:} 18 out of 20.\\
			Academic performance in curricular part got him the \textbf{2013/2014 
			Multicert School Award}\footnote{See 
			\url{https://www.multicert.com/en/about-us/universities/}.}.
    \item {\bfseries University of Coimbra}, Coimbra, Portugal.\\
			\emph{Licenciatura} (roughly a 5 year pre-Bologna BSc) in Informatics 
			Engineering, Fall 2001--Fall 2006.
  \end{itemize}
\end{topic}

\begin{topic}{Professional Experience}
  \begin{itemize}[leftmargin=*]
    \item {\bfseries AnubisNetworks}---Lisbon, Portugal.\\
      Software Engineer (multi-class), June 2007 -- April 2013.
      \begin{itemize}
				\item {\scshape From October 2012} -- Moved to the Operations team. 
					Handled SpamAssassin management and performance tunning. Developed 
					software to partially automate some management tasks, as well as spam 
					statistics collection and analysis.\\
					{\scshape Requirements}: SpamAssassin, Perl, Apache, RRDTool.
				\item {\scshape From June 2011} -- Development of \emph{myFamily} and \emph{Web 
					Protection Service} web content filtering services. Focus on user 
					interface and authentication.\\
					{\scshape Requirements}: C{}\verb!++!, HTML, CSS, HTTP, tcpflow.
				\item {\scshape From May 2009} -- Participated in the development of a distributed 
					architecture for MPS. The purpose was to allow multi-node 
					installations, in order to meet the scalability requirements needed by 
					carriers and other large customers.\\
					{\scshape Requirements}: Perl, Postfix, LDAP.
				\item {\scshape From June 2007} -- Implemented the Reports 
					module of the \emph{Mail Protection 
					System}\textsuperscript{\texttrademark} appliance (MPS)---Anubis' 
					flagship product. Redesigned the latter's database to support that 
					module---and to begin separating business logic from underlying system 
					actions. Also became the \emph{de facto} maintainer of the web 
					interface, during this period.\\
					{\scshape Requirements}: Perl, Apache, MySQL.
      \end{itemize}
  \end{itemize}
\end{topic}

\begin{topic}{Technical Skills}
  {$\rhd$\ \ \bfseries \scshape Programming Languages}
  \begin{itemize}[leftmargin=*]
    \item{\bfseries Imperative}: \emph{proficient}: C, Perl; \emph{experienced}: C{}\verb!++!, Python;
      \emph{basic skills}: Java, Javascript/JQuery, Vim script.
    \item{\bfseries Functional}: \emph{basic skills}: Common Lisp.
    \item{\bfseries Markup}: \emph{proficient}: HTML, CSS, \LaTeX, Markdown; \emph{basic skills}: UML.
  \end{itemize}
  {$\rhd$\ \ \bfseries \scshape Operating Systems}
  \begin{itemize}[leftmargin=*]
		\item{\bfseries GNU/Linux}: Began experimenting different distributions in 
			2000; became a regular user circa 2005.
      \begin{itemize}
				\item Proficient \textsc{Arch Linux} user (current distribution of choice for personal computers).
				\item Experienced with \textsc{Debian} and \textsc{Ubuntu Server} (mainly for small-network servers).
      \end{itemize}
		\item{\bfseries Microsoft Windows}: Last version regularly used was XP 
			(SP3). Nowadays, occasionally fires up Win 10, to play the occasional 
			computer game.
  \end{itemize}
	\newpage % obviously, comment this if unneeded
  {$\rhd$\ \ \bfseries \scshape Miscellaneous Software}\\[1em]
	\begin{minipage}{0.5\textwidth}
		\begin{itemize}
			% see comment at header for \phantom...
			\item{\bfseries Text Editors}: Vim (proficient).\\\phantom{xpto}
			\item{\bfseries Mathematics}: GNUplot, SageMath, 
				\texttt{libgmp}.
		\end{itemize}
	\end{minipage}%
	\begin{minipage}{0.5\textwidth}
		\begin{itemize}
			\item{\bfseries Network}: Open\{SSH,SSL\}, tcpflow, iptables, RRDTool, GnuPG.
			\item{\bfseries Databases}: MySQL, SQLite, PostgreSQL.
		\end{itemize}
	\end{minipage}
\end{topic}

\begin{topic}{Languages}
  {\hspace*{2em} \slshape \small \textbf{Common European Framework of Reference for Languages} 
	(\href{http://en.wikipedia.org/wiki/Common\_European\_Framework\_of\_Reference\_for\_Languages#Common\_reference\_levels}{CEFRL}) 
	levels given for non-native idioms.}
  \begin{itemize}[leftmargin=*]
    \item {\bfseries Portuguese}: native language. (\emph{Note: European Portuguese})
    \item {\bfseries English}: fluent, both written (C1) and spoken (C1).
    \item {\bfseries Spanish}: basic writing skills (A2) but fluent speaker (B2).
  \end{itemize}
\end{topic}

\begin{topic}{Additional Information}
  {\hspace*{2em} \itshape Informal Education / Professional Development}
  \begin{itemize}[leftmargin=*]
    \item {\bfseries Cryptography I}, Stanford University, 2012.\\
      Coursera Online Platform -- \url{http://www.crypto-class.org}. \emph{Grade:} 98\%.
    \item {\bfseries Curso Especial de Empresa}, (employer-sponsored Spanish language training), 2008--2009.\\
			Instituto Cervantes de Lisboa. \emph{Grade:} Sobresaliente (9 out of 10).
  \end{itemize}
  {\hspace*{2em} \itshape Miscellanea}  
  \begin{itemize}[leftmargin=*]
    \item{Outdoor sports enthusiast---mainly surf and football.}
    \item{Enjoys recreational---and often not-so-recreational---mathematics,
      as well as physics (and chess).}
    \item{Participated in the 2003 and 2004 editions of \emph{Minho Campus
			Party} (Portugal's first large scale LAN party!).}
		\item Writes at \url{https://erroneousthoughts.org/blog}. Paraphrasing the 
			Nobel prize winning physicist Richard Feynman: \emph{to explain, one must 
			first understand}. Writing being mostly about the former, it provides a 
			strong incentive to put in the effort needed to achieve the latter. 
			Besides, writing helps one to become a better---and faster!---reader.
    \item The above notwithstanding, he chronically tends to accumulate more books than 
      what he is able to read.
    \item{Full Portuguese driving license---Category B, passenger light
      vehicles.}
  \end{itemize}
\end{topic}
\end{document}

% vim: spell spelllang=en
