\documentclass[a4paper,12pt]{letter}

% --- Set up fonts to use ---
\usepackage[bitstream-charter]{mathdesign}
\DeclareSymbolFont{usualmathcal}{OMS}{cmsy}{m}{n}
\DeclareSymbolFontAlphabet{\mathcal}{usualmathcal}
\usepackage{fontspec}
\setmainfont[Scale=0.9, Ligatures=TeX]{Charis SIL}
\defaultfontfeatures[Charis SIL]{Script=latn}
%\usepackage{microtype}                                    % see README
% --- END code for fonts ---
% --- Additional packages here ---
\usepackage[UKenglish]{babel}                               % replace UKenglish with portuguese or other lang
\usepackage{endnotes}                                       % comment if not needed
\renewcommand{\theendnote}{\Roman{endnote}}                 % Endnotes in Roman numerals (arabic if for footnotes)
% --- END additional packages ---
% --- Page layout ---
\usepackage[top=2.5cm, bottom=2.5cm, left=3.0cm, right=3.0cm]{geometry}
\pdfpagewidth 210mm
\pdfpageheight 297mm
% no line numbering
\pagestyle{empty}
% avoid having the closing too ``centered''
\longindentation=230pt
% --- END Page layout ---
% --- Commands (new or redefine) ---
% --- END commands ---
% --- Includes ---
%% general purpose
\newcommand{\ts}{\textsection}
\newcommand{\emd}{\textemdash}                             % see [1]

%% prevent figures that end up on the last empty 
%% page of a document from being forcibly centred 
%% on the middle of the page.
\makeatletter
\setlength{\@fptop}{0pt}
\makeatother

% [1] - When inserting an em-dash like this: --- any adjacent words 
%       are NOT HYPHENATED. Using \textemdash (or the newly defined 
%       shortcut)fixes this issue.

%% --- math packages (in ArchLinux they're all part of TeXLive) ---

\usepackage{amsmath}   % has to come before eulervm, or things bork!
\usepackage{amsthm}    % required for proof environ (et al.)

% --- END math packages ---

% math stuff
\newcommand{\eqv}{\Leftrightarrow}
\newcommand{\pvec}[1]{\overrightarrow{#1}}
\newcommand{\npvec}[1]{\norm{\pvec{#1}}}
\newcommand{\mb}[1]{\ensuremath \mathbb{#1}}
\newcommand{\mc}[1]{\ensuremath \mathcal{#1}}
\newcommand{\f}[2]{\ensuremath \frac{#1}{#2}}

\providecommand{\abs}[1]{\left \lvert #1 \right \rvert}
\providecommand{\norm}[1]{\left \lVert #1 \right \rVert}

\providecommand{\pth}[1]{\left ( #1 \right ) }
\providecommand{\spth}[1]{\left [ #1 \right ] }
\providecommand{\cpth}[1]{\left { #1 \right } }

% make the optional theorem name bold
\makeatletter
\def\th@plain{%
  \thm@notefont{}% same as heading font
  \itshape % body font
}
\makeatother

% math theorem environments
\newtheorem{thm}{Theorem}[section]
\newtheorem{corollary}[thm]{Corollary}
\newtheorem{lemma}[thm]{Lemma}
\newtheorem{definition}[thm]{Definition} % 'def' cannot be used as environ name

\theoremstyle{remark}
\newtheorem{remark}[thm]{Remark}

\renewcommand\qedsymbol{$\blacksquare$}  % for proofs to end with a black *filled* square (requires amssymb)

%
% math hacks
%

% scalable middle bar; see 
% http://tex.stackexchange.com/questions/5502/how-to-get-a-mid-binary-relation-that-grows
\newcommand{\relmiddle}[1]{\mathrel{}\middle#1\mathrel{}}

% it's a cheat but this makes spacing less wrong around quantifiers
\DeclareMathOperator{\Nexists}{\nexists}
\DeclareMathOperator{\Exists}{\exists}
\DeclareMathOperator{\Forall}{\forall}

% --- END Includes ---

% --- Letter specific info ---
% for letters, I prefer to set the date explicitly
\date{February 31, 2013} 
\address{
  22nd Baking Street\\
  2099--101 Acme City\\
  LooneyTuneLand
}
\signature{Your Truly's Name}
% --- END PREAMBLE ---

\begin{document}
\begin{letter}{
    R. MaDillo\\
    European Parliament\\
    Rue Wiertz\\
    Altiero Spinelli 07F343\\
    B-1047 Bruxelles 
  }
  %\opening{Ex.\textsuperscript{o} Doutor R. Madillo,} % -- Portuguese introduction style
  \opening{Dear Dr. R. MaDillo,}
{
  \begin{center}
    \textbf{\textsc{Subject of your letter here}}\\[2em]
  \end{center}
}
% The ``extra'' braces above and below are required, if you want to use 
% paragraph first line indentation (and thus not use skip space between 
% paragraphs) as is customary when writing in Portuguese.
%
% Comment the two \setlength commands below if this *undesired*.
{
  \setlength{\parindent}{0.5cm}
  \setlength{\parskip}{0pt plus 0.1pt}

  Lorem ipsum dolor sit amet, consectetur adipiscing elit. Donec bibendum 
  suscipit odio, id dapibus ipsum placerat vitae. Quisque vitae magna vel ante 
  semper dapibus sit amet eu elit. Fusce cursus sodales eros, non porta urna 
  bibendum et. Ut mauris risus, malesuada sit amet semper ac, molestie in 
  tellus. Nam imperdiet metus sed massa faucibus sed vulputate risus laoreet. 
  Aenean tincidunt, purus ac euismod lobortis, tortor lacus iaculis eros, non 
  egestas orci massa in ligula. Cras bibendum suscipit lacus nec mollis. Nullam 
  neque erat, molestie ac bibendum eget, molestie non nunc. Morbi quis orci 
  augue. Pellentesque facilisis, erat sed pellentesque rutrum, orci turpis 
  dictum mi, sit amet rutrum nibh quam in velit. Vestibulum semper, est eget 
  blandit pellentesque, nisi elit venenatis diam, id vestibulum velit leo quis 
	orci.\endnote{Some endnote-ing text.} 
  
  In dui magna, posuere eget, vestibulum et, tempor auctor, justo. In ac felis 
  quis tortor malesuada pretium. Pellentesque auctor neque nec urna. Proin 
  sapien ipsum, porta a, auctor quis, euismod ut, mi. Aenean viverra rhoncus 
  pede. Pellentesque habitant morbi tristique senectus et netus et malesuada 
  fames ac turpis egestas. Ut non enim eleifend felis pretium feugiat. Vivamus 
  quis mi. Phasellus a est. Phasellus magna.
}

  \closing{Yours sincerely,}
%
  \ps{P.\ S.: Some more latin should go here.}
  \encl{A vacation postcard!}

  % --- show endnotes ---
  % comment if not using the endnotes package
  \vspace{1cm}
  \begingroup
  %\addcontentsline{toc}{section}{Notes}
  \def\enotesize{\small}
  \def\enoteheading{\Large \textsc{Notes}}
  \theendnotes
  \endgroup
  % --- end endnotes ---
\end{letter}
\end{document}

% ALTERNATIVE CLOSING: if you want to send a letter that will not be 
% printed, you can add an image with your signature replacing the \closing 
% with this:
%
%  \closing{Yours sincerely,\\
%    \fromsig{\includegraphics[scale=0.5]{signature.png}}\\
%    \fromname{Your Truly's Name}
%  }
%
% NOTE: If you use this, you have to remove the \signature, in the preamble.
