\documentclass[a4paper,12pt,dvipsnames*]{article}
\input{inputs/preamble}

\title{
Write, or wrong?
}
\author{
  Óscar Pereira
}

% This negative space is required to avoid having too much space between the
% title of the document and the title of the first section.
\date{\vspace{-4em}}
% --- END PREAMBLE ---

\begin{document}

\maketitle
\fnnosym{\emph{Date:} \today. \emph{Contact information:} \texttt{\{https://, oscar@\}randomwalk.eu}.}

% The setting in this \begin group change hyperref settings to make all of the
% TOC (entries + page numbers) clickable, but in black (instead of in the color
% of links).
%
% Also, if a fancy page header is set, the stuff on the line that contains the
% \tableofcontents command is needed to, if the TOC occupies more than one
% page, avoid having that fancy header show up from page 2 (of the TOC)
% onwards.
\begingroup
\hypersetup{linkcolor=black}
{\pagestyle{plain}\tableofcontents}
% {\pagestyle{plain}\tableofcontents\cleardoublepage}
% NB: about the two lines above (one commented), the \cleardoublepage causes the contents of the document to be placed in a new page. This is what I want for report, but not for essay. It remains to be seen if removing that \cleardoublepage is enough, even when the TOC has more then one page (cf. remarks above, before \begingroup).
\endgroup

\section{Introduction}
  Something less technical about Feynman~\cite{Feynman74} and Sagan~\cite{Sagan}.

\phantomsection
\addcontentsline{toc}{section}{References}
\bibliographystyle{sane}
\bibliography{sources}
\end{document}

% vim: spell spelllang=en
