\documentclass[a4paper,10pt,twocolumn]{article}

%\usepackage[portuguese]{babel}     % i18n
\usepackage[xetex]{graphicx}        % this borks if included after next line, comment if unneeded
\usepackage{xltxtra,xunicode}        
\usepackage{url}

% --- to add other fonts to document --- 
\usepackage{fontspec}                    % only needed if using custom fonts
\defaultfontfeatures{Mapping=tex-text}	 % needed otherwise ligatures and quotes get beserk

% %%%
% and now load the fonts proper
% *** font packages needed in ArchLinux ***
% yaourt -S ttf-adobe-fonts
% yaourt -S monaco-linux-font
% %%%
\setromanfont [Ligatures={Common}, BoldFont={Minion Pro Bold}, ItalicFont={Minion Pro Italic}]{Minion Pro}
\setmonofont  [Scale=0.8]{Monaco}
% --- end code for other fonts ---

% --- additional packages here ---
\usepackage{amsfonts}
\usepackage{subfig}
% --- END additional packages ---

% --- Page layout ---
\usepackage[top=2.5cm, bottom=2.5cm, left=1.5cm, right=1.5cm]{geometry}
\pdfpagewidth 210mm
\pdfpageheight 297mm

\setlength{\columnsep}{20pt}

% avoid having footnote text too small
\renewcommand{\footnotesize}{\small}

% the hyperref package if used, MUST BE THE LAST ONE included
\usepackage[pagebackref=true,bookmarks=true,colorlinks=true,linkcolor=blue,citecolor=cyan,linktocpage=true]{hyperref}
% --- END Page layout ---

% --- Commands (new or redefine) ---
\newenvironment{definition}[1][Definition]{\begin{trivlist}
\item[\hskip \labelsep {\bfseries #1}]}{\end{trivlist}}

% shortcut for \textsection
\newcommand{\ts}{\textsection}
% --- END commands ---

% --- Title page ---
\title{\normalfont\Huge\textbf{A skel for \LaTeX}}
\author{Óscar Pereira\\\texttt{my.email@provider.tld}}
\date{}
% --- END title page ---

% --- END PREAMBLE ---

\begin{document}
\maketitle              % need full-width title

%\tableofcontents        % if needed

% --- D.J. Bernstein style document id's ---
% command I use: head /dev/urandom | sha1sum
% see http://cr.yp.to/bib/documentid.html
%
{\let\thefootnote\relax\footnote{Date of this document: September 2010. Permanent ID:
\mbox{\textbf{\texttt{a83b82766cb922e7507489df3b91dce1710bd3b1}}} \\ \\The
author hereby places this document in the public domain.}
\addtocounter{footnote}{-1}}
% --- END document id's ---

\begin{abstract}
	Some tips for \LaTeX\ template.
\end{abstract}

\section{Styles, et al.}
Here I’ll describe a way of installing \LaTeX styles, BiBTeX styles, and a couple 
of other things in a straightforward way. I think this should work with most 
*nix systems.

The first thing to do, is discover where whatever you want to install should be 
installed. And the ideal way to do this is using a tool named kpsewhich, which 
should get installed when you install LaTeX. It can be used to do a lot of 
things\\(\verb+$ kpsewhich --help+), but the one we’re interested in here, 
location of styles, uses the \verb+--show-path NAME+ option. The list of allowed names 
is part of the output of the \verb+--help+ option. So for instance, to discover where to place 
BiBTeX style files (*.bst), run:
\begin{verbatim}
kpsewhich --show-path bst
\end{verbatim}
This will output a list of locations where BiBTeX style files are searched for. 
So if you have a file called mystyle.bst, create a folder named ``mystyle" in the 
appropriate location (I use \verb+/home/user/texmf/bibtex/bst/+), and put mystyle.bst 
inside the folder you just created. Then run \verb+$ texhash .+ (don’t forget the 
dot!) from the ``appropriate location" folder you used. And you're done!

\section{Math Stuff}
The black QED square can be done by defining this in the preamble:
\begin{verbatim}
\newcommand{\qed}{\hfill \mbox{\raggedright \rule{.07in}{.07in}}}
\end{verbatim}
Some other environments that can be useful are the following:
\begin{verbatim}
\newtheorem{theorem}{Theorem}[section]

\newtheorem{lemma}[theorem]{Lemma}

\newtheorem{proposition}[theorem]{Proposition}

\newtheorem{corollary}[theorem]{Corollary}

\newenvironment{proof}[1][Proof]{\begin{trivlist}
\item[\hskip \labelsep {\bfseries #1}]}{\end{trivlist}}

\newenvironment{definition}[1][Definition]{\begin{trivlist}
\item[\hskip \labelsep {\bfseries #1}]}{\end{trivlist}}

\newenvironment{example}[1][Example]{\begin{trivlist}
\item[\hskip \labelsep {\bfseries #1}]}{\end{trivlist}}

\newenvironment{remark}[1][Remark]{\begin{trivlist}
\item[\hskip \labelsep {\bfseries #1}]}{\end{trivlist}}
\end{verbatim}

Finally, spacing: In a ``math" environment, LaTeX ignores the spaces you type 
and puts in the spacing that it thinks is best. LaTeX formats mathematics the
way it’s done in mathematics texts. If you want different spacing, LaTeX 
provides the following four commands for use in math mode:

\begin{verbatim}
\; - a thick space

\: – a medium space

\, – a thin space

\! – a negative thin space 
\end{verbatim}
\bibliographystyle{is-abbrv}
% \bibliography{bib_file}     % for BiBTeX file
\end{document}

