\usepackage[UKenglish]{babel}                              % replace UKenglish with portuguese or other lang
\usepackage{fontspec}                                      % needed for non-ascii chars to display correctly

\usepackage{sfmath}                                        % math in sans serif font
\usepackage{tikz}                                          % to draw pictures!
\usepackage{marvosym}                                      % for \Smiley, and LOTS of other things

% Arev is a great font for presentations. However, it cannot display some
% characters, namely € (euro), «, » and §. If these are really necessary, 
% then comment the \usepackage line, and...
\usepackage{arev}
% ... uncomment the two lines below.
% \renewcommand{\rmdefault}{\sfdefault}
% \renewcommand{\emph}[1]{\textbf{#1}}

% For fancy title with \pagestyle{empty} (for no page num on title slide)
\newcommand{\mytitle}[1]{%
	\par\noindent\colorbox{blue!60}%
	{\parbox{\dimexpr\textwidth-2\fboxsep\relax}%
	{\setlength{\baselineskip}{2\baselineskip}%
		\centering\textcolor{white}{\textbf{#1}}}}}

% In slides, emphasis means italic AND bold face.
\renewcommand{\emph}[1]{\textbf{\textit{#1}}}
% Align with \fadeTo (easier for vim-motions when changing numbers)
\newcommand{\fadeFr}[1]{\fadeFrom{#1}}

\style{slide.size=130}                                     % use big fonts and images
\style{foot.text.bot.right=\theslidenum/\lastslide}        % slide numbers are ugly... but necessary
\linespread{1.5}                                           % for lines bigger than slide width, 1.5 space

% Different kinds of (symbol) bullets
\newlength{\bulletIndentLen}
\setlength{\bulletIndentLen}{0.5em}
\newcommand{\bulletA}{\hspace{\bulletIndentLen}$\ast$ }
\newcommand{\bulletB}{\hspace{\bulletIndentLen}$\bullet$ }
\newcommand{\bulletD}{\hspace{\bulletIndentLen}$\diamond$ }
\newcommand{\bulletT}{\hspace{\bulletIndentLen}$\triangleright$ }

% Numbered bullet:
% \bulletN{<number>} text...
\newcommand{\bulletN}[2]{%
	\hspace{0.5em}$#1.$ #2}

\newcommand{\topic}[1]{%
	\Banner{#1}}

%---------------------
% --- CUSTOM STUFF ---
%---------------------
%-------------------------
% --- END CUSTOM STUFF ---
%-------------------------
