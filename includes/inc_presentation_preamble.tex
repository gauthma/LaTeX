\usepackage[UKenglish]{babel}                               % replace UKenglish with portuguese or other lang
\usepackage{fontspec}                                       % needed for non-ascii chars to display correctly

\usepackage{sfmath}
\usepackage{tikz}
\usepackage{marvosym}                                       % for \Smiley, and LOTS of other things

% Arev is a great font for presentations. However, it cannot display some
% characters, namely € (euro), «, » and §. If these are really necessary, 
% then comment the \usepackage line, and...
\usepackage{arev}
% ... uncomment the two lines below.
% \renewcommand{\rmdefault}{\sfdefault}
% \renewcommand{\emph}[1]{\textbf{#1}}

% For fancy title with \pagestyle{empty} (for no page num on title slide)
\newcommand{\mytitle}[1]{%
	\par\noindent\colorbox{blue!60}%
	{\parbox{\dimexpr\textwidth-2\fboxsep\relax}%
	{\setlength{\baselineskip}{2\baselineskip}%
		\centering\textcolor{white}{\textbf{#1}}}}}

% In slides, emphasis means italic AND bold face.
\renewcommand{\emph}[1]{\textbf{\textit{#1}}}

% \style{slide.size=90}                                     % use big fonts and images
\style{foot.text.bot.right=\theslidenum/\lastslide}         % slide numbers are ugly... but necessary

% Different kinds of (symbol) bullets
\newcommand{\bulletA}[1]{ %
	\hspace{0.5em}$\ast$ #1}
\newcommand{\bulletB}[1]{ %
	\hspace{0.5em}$\bullet$ #1}
\newcommand{\bulletD}[1]{ %
	\hspace{0.5em}$\diamond$ #1}
\newcommand{\bulletT}[1]{ %
	\hspace{0.5em}$\triangleright$ #1}

% Numbered bullet:
% \bulletN{<number>} text...
\newcommand{\bulletN}[2]{ %
	\hspace{0.5em}$#1.$ #2}

\newcommand{\topic}[1]{ %
	\Banner{#1}}

% --- math packages (in ArchLinux they're all part of TeXLive) ---

\usepackage{amssymb}   % for blackboard math (automagically loaded in article)
\usepackage{amsmath}   % has to come before eulervm, or things bork!
\usepackage{amsthm}    % required for proof environ and remark theoremstyle (et al.)
% --- END math packages ---

% Uncomment if needed.
% % math stuff
\newcommand{\eqv}{\Leftrightarrow}

\newcommand{\pvec}[1]{\ensuremath{\overrightarrow{#1}}}
\newcommand{\npvec}[1]{\ensuremath{\norm{\pvec{#1}}}}

\newcommand{\mb}[1]{\ensuremath{\mathbb{#1}}}
\newcommand{\mc}[1]{\ensuremath{\mathcal{#1}}}

\providecommand{\abs}[1]{\ensuremath{  \left \lvert #1 \right \rvert } }
\providecommand{\norm}[1]{\ensuremath{ \left \lVert #1 \right \rVert } }

\providecommand{\pth}[1]{\ensuremath{  \left ( #1 \right ) } }
\providecommand{\spth}[1]{\ensuremath{ \left [ #1 \right ] } }
\providecommand{\cpth}[1]{\ensuremath{ \left \{ #1 \right \} } }

%
% math hacks
%

% scalable middle bar; see 
% http://tex.stackexchange.com/questions/5502/how-to-get-a-mid-binary-relation-that-grows
\newcommand{\relmiddle}[1]{\mathrel{}\middle#1\mathrel{}}

% it's a cheat but this makes spacing less wrong around quantifiers
\DeclareMathOperator{\Nexists}{\nexists}
\DeclareMathOperator{\Exists}{\exists}
\DeclareMathOperator{\Forall}{\forall}


% math theorem environments
\newtheoremframe{theorem}{Theorem}
\newtheoremframe{corollary}{Corollary}
\newtheoremframe{lemma}{Lemma}
\newtheoremframe{definition}{Definition} % 'def' cannot be used as environ name
\theoremstyle{remark}
\newtheoremframe{remark}{Remark}
